. env is used to keep all our secrets like password and other user data
also we use the . gitignor to ignore our .env and also to ignor our module

mongoose is a frame work for mongodb 
schema must be defind mongo can take in date, number string buffer mixed object id interger nboolan array and the rest
 schema for a blog, 
 schema in mongoose is mainly all baiut schema
 schema is used to mordefie user info and alos keep all user data to be used leter  in your data
 bulidong a schema for goal project schema has a templet for dev its always in a folder called modeels
 inside of d folder u create a file for your project
 after creating  a file for your schemayou put in d timestamp for when the user crearted d doc and when it was updated
 also u have to export ypur schema for you to intract wit ur schema u use d goal  . anytging u wanrto to do 
  oce a model is been created you hav to  create a function dat connects wit d model
  after u are done wit ur model u now create a controller datb controls ur schema
cretaw a route for ur router after importing ur 

 we now set up our function in postman 
 middel ware are func dat run on server bet req and res

 /// create a pack. json to track evertun
  den a aname file to watch wat wwe doing
  2. connection of app wit database
  3. d mongodb is used to connect ur app wit it
  4. we connect wit await.connection
  5. u create a database name
 6. u den create a lisiting to d project
 7. also if connection is wrong we kil d server n exist
 all we didi yedterfayn is called mvc archirecture
 mvc =  model is were we build d schema

 schema is how will be modeling our apptimestamp is neede for every schema u created.
 alos timestamp assit wit data agrigaton
 after buildin ur model u  export 
 // C = controller. 
  from model we move to controllers 
  controllers is mainly were all  func is written 

  after u done wit d model we control dem in d controller den we now route dem, like giving dem direction on were to go.

  after it we now import it into our app.js 
  so d req and res in our code is called request ansd respond

  we use cors to be able to use at other servers
 

